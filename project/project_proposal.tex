\documentclass{article}
\usepackage[letterpaper, total={6.5in, 10in}]{geometry}
\usepackage{rotating}
\usepackage{pgfgantt}

\title{E599 Intro to Haptics\\Project Proposal}

\author{Wesley Liao}

\begin{document}

\maketitle

\section{High Level Deliverables}
\begin{enumerate}
  \item Demonstrate table slider motion and actuation functionality through
    simple single-user haptic environments.
    This will require implementing motor control and position feedback, and
    tuing the physical moving components of the slider to function smoothly.

  \item With the underlying slider capability implemented, extend functionality
    to a two participant tracking task with visualizations.
    The task is conceptualized as two people each with a rod carrying an egg.
    The subjects must suspend the egg between the ends of the rods, and
    balancing the pressure they apply together to neither break nor drop the egg.
    While maintaining appropriate force to keep the egg suspended, the
    participants are tasked with tracking a sinusoidal trajectory. Task
    performance is measured by how well they track the trajectory and the number
    of eggs dropped.
    Both an individual and dyad version of the tracking task will be needed. The
    individual version should be the same as the dyad version, but with the
    second user simulated as a constant force or a simple deterministic policy.

  \item The final component is to develop an experiment protocol for running
    sets of trials with participants and collecting data. Experiments would
    ideally compare individual and dyad performance as well as performance
    between different pairs of individuals.
\end{enumerate}

\section{Implementation Requirements}
\subsection*{Numbering Scheme}
A.X.X.X.c
\begin{enumerate}
\item[A] First capital letter denoting hardware or software.
\item[X] Following numerals denoting levels of sub-components.
\item[c] Final lower case letter denoting a testable feature required for a sub-component.
\end{enumerate}

\subsection*{Hardware Components}
    \begin{enumerate}
    \item[H.1] Dyad Slider
        \begin{enumerate}
        \item[H.1.a] Will be usable by two seated subjects, sliding along a single axis in a comfortable range of motion by use of slider handles.
        \item[H.1.b] Capable of measuring slider position and force applied to handles along slider axis.
        \item[H.1.c] Capable of applying sufficient force to the slider to simulate dynamics.
        \item[H.1.d] Applied force should be limited to a safe level of maximum output.
        \end{enumerate}
    \item[H.2] Subject Screens
        \begin{enumerate}
        \item[H.2.a] Each screen should be clearly visible to one subject and invisible to the other subject while both are seated in use of the slider.
        \end{enumerate}
    \end{enumerate}

\subsection*{Software Components}
    \begin{enumerate}
      \item[S.1] Input and Output
      \begin{enumerate}

        \item[S.1.1] Motor Control
          \begin{enumerate}
          \item[S.1.1.a] Will control motor torque as a function of slider state and task state.
          \end{enumerate}

        \item[S.1.2] Slider State
          \begin{enumerate}
          \item[S.1.2.a] Will measure encoder position.
          \item[S.1.2.b] Will estimate velocity and acceleration of slider.
          \item[S.1.2.c] Will measure applied force to slider handles.
          \end{enumerate}

        \item[S.1.3] Data Collection
          \begin{enumerate}
            \item[S.1.3.a] Data will be formatted as human-readable, comma-deliminated text.
            \item[S.1.3.b] Motor state, slider state, and task and experiment state will be recorded.
          \end{enumerate}

      \end{enumerate}

    \item[S.2] Reference Tracking Task
        \begin{enumerate}

        \item[S.2.1] Visualization
            \begin{enumerate}
            \item[S.2.1.a] Capable of displaying consistent 60 Hz or higher framerate.
            \item[S.2.1.b] Capable of separate visualizations presented on each subject screen.
            \item[S.2.1.c] Visualizations will be composed of reference trajectories and cursors.
            \item[S.2.1.d] Cursor positions in the visualization will be set by a function of time, task state, and slider state.
            \item[S.2.1.e] Reference trajectories will be defined as a position function of time.
            \end{enumerate}

        \item[S.2.2]  End condition
            \begin{enumerate}
            \item[S.2.2.a] Capable of triggering the trial end by some function of task state and input.
            \item[S.2.2.b] Can be set to require trigger condition to be true for an amount of time before triggering end of trial.
            \item[S.2.2.c] Can be set to either reset the accumulated time to zero or continue counting if the trigger condition becomes false before enough time is elapsed to trigger end of trial.
            \item[S.2.2.d] Will return a exit code to be used by experiment structure.
            \end{enumerate}

        \end{enumerate}

       
    \item[S.3.] Experiment Structure
        \begin{enumerate}
        \item[S.3.1] Text Instructions
            \begin{enumerate}
            \item[S.3.1.a] Will display text to inform subjects.
            \item[S.3.1.b] Can be set to display on either subject screen individually or both screens simultaneously.
            \item[S.3.1.c] Accepts subject input to proceed to next step of trial or return to previous step after a set minimum viewing time.
            \item[S.3.1.d] Can be set to automatically advance to next step after a timeout, or wait indefinitely for input.
            \end{enumerate}

        \item[S.3.2] Task Trials
            \begin{enumerate}
            \item[S.3.2.a] Will have methods of prescribing sequences and
                control flow of multiple trials based on exit codes of individual trials.
            \end{enumerate}
        \end{enumerate}

    \end{enumerate}

\newpage
\begin{sidewaysfigure}

\section{Timeline}
\begin{ganttchart}[
  vgrid,
  time slot format=isodate,
  expand chart=10in,
  y unit chart=0.25in,
  bar/.append style={fill=black},
  bar incomplete/.append style={fill=white},
  progress label text=\relax,
%  today={\the\year-\the\month-\the\day},
  ]{2020-03-02}{2020-04-28}

  \gantttitlecalendar{month=name, week, day} \\

  \ganttbar[inline, progress=0]{Spring Break}{2020-03-14}{2020-03-22} \\

  \ganttgroup[inline]{Hardware Components}{2020-03-02}{2020-03-13} \\
    \ganttbar[progress=0]{Slider Motor Refit}{2020-03-02}{2020-03-06} \\
    \ganttbar[progress=0]{Wire Guide Improvements}{2020-03-09}{2020-03-13} \\

  \ganttgroup[inline]{Software Components}{2020-03-23}{2020-04-17} \\
    \ganttbar[progress=0]{Slider State Input}{2020-03-23}{2020-03-25} \\
    \ganttbar[progress=0]{Motor Control Output}{2020-03-25}{2020-03-27} \\
    \ganttbar[progress=0]{Data Output}{2020-03-30}{2020-03-31} \\
    \ganttbar[progress=0]{Haptic Demos}{2020-04-01}{2020-04-03} \\
    \ganttbar[progress=0]{Dyad Task}{2020-04-06}{2020-04-10} \\
    \ganttbar[progress=0]{Individual Task}{2020-04-13}{2020-04-15} \\
    \ganttbar[progress=0]{Experiment Structure}{2020-04-16}{2020-04-17} \\
%    \ganttbar[progress=0]{Software Validation}{2020-04-20}{2020-04-24} \\

    \ganttgroup[inline]{Experiment Design}{2020-03-02}{2020-04-24} \\
      \ganttbar[progress=0]{Procedure Draft}{2020-03-02}{2020-03-13} \\
      \ganttbar[progress=0]{Procedure Revisions}{2020-03-23}{2020-04-15} \\
      \ganttbar[progress=0]{Procedure Testing}{2020-04-20}{2020-04-24} \\

\end{ganttchart}
\end{sidewaysfigure}
\end{document}
