\documentclass{article}
\usepackage[letterpaper, total={6.5in, 9in}]{geometry}

\title{E599 Intro to Haptics\\Lab 1: Haptic Rendering}

\author{Wesley Liao\\Collaborators: Britain Taylor}

\begin{document}

\maketitle

\section{Introduction}

Haptic interfaces provide a sense of force-feedback to the user, in respect to a
simulated system or visualization.
The goal of this lab is to implement convincing policies for rendering motor torques in
simple simulated environments.

\section{Methods}

\subsection{Hardware}

  The base hardware is a version 3 Hapkit [1].
  Hapkit is a 3D-printable single degree-of-freedom (DOF) haptic interface with a
  revolute paddle, driven by a small DC motor.

  The electrical control system is a custom Arduino-based board with an
  integrated PWM motor driver and magnetic angle sensor.

\subsection{Software}

  A starting point for the software was provided in two parts:
  an Arduino program for multiple modes of motor control and
  a pygame visualization written in python.
  The control board program and the python visualization communicate by serial USB.

  The Arduino program is structured with a main motor control loop that calls a
  policy function to determine the output duty cycle for the motor.
  The policy functions accept arguments of paddle angle, angular velocity, and
  angular acceleration, returning a signed relative motor torque ranging from -1 to 1.

  In these policy functions each of the different visualizations is implemented.

\subsection{Visualizations}

  We implemented motor output policies for seven visualizations:
  Flat, Wall, Wall with Impact Vibrations, Valley, Hillock, and Notch.
  In each visuialization a cursor moves along a surface.

  \subsubsection{Flat}

    A plane with no features.
    In this visualization we simulated the cursor mass and intertia.
    Torque is calculated using the estimated angular acceleration multiplied by an arbitrary mass.

  \subsubsection{Wall}

    A flat plane with a sudden wall, simulated as a spring with viscous damping.
    Spring constant set to 0.1 relative torque per degree.
    Damping coefficient of 0.1 relative forque seconds per degree.

  \subsubsection{Wall with Impact Vibrations}

    Visually identical to the wall visualization, but to simulate a more
    realistic sensation of colliding with a rigid wall, vibrations were
    layered on top of the spring contact force and damping.
    The vibration function is a decaying sinusoid, demonstrated by Okamura et. al. [2]
    Vibration initial amplitude is determined by the angular velocity at time of
    impact, decaying until the cursor leaves the wall boundary.

  \subsubsection{Valley}

    The inside lower-half of a circular ``half pipe''.
    In this scenario, gravity pulls the cursor to the bottom of the valley.
    The torque is set to the component of the downward gravitational force that is
    tangental to the valley surface, multiplied by an arbitrary constant.

  \subsubsection{Hillock}

    The cursor rolls on the outside top-half of a circular surface.
    Hillock simply returns the Valley visualization torque value multiplied by negative 1.

  \subsubsection{Notch}

    The cursor moves along flat surface with a single detent.
    A constant torque is applied if the cursor is within the range of the
    detent, driving the cursor to the detent center. A small deadband of no
    torque was added at the bottom of the detent to prevent jittering.

  \subsubsection{Bumpity}

    A series of notches covering the full width of the visualization, with no
    space seperating them.
    To implement Bumpity, the cursor position was resolved to a multiple of the
    notch width, then passed to the Notch visualization.

\subsection{Implementation Foundations}

  \subsubsection{Motor Deadband}
    Friction in the hapkit causes the paddle to not move when lower motor voltages
    are appiled. To counteract this, we added a deadband to the motor command.

    The control board motor driver command is set by a direction bit and an
    8-bit (0 to 255) PWM duty cycle.

    We gradually increased the  motor duty cycle to determine when the motor was
    just barely able to overcome the friction inherent in the hapkit, arriving at a duty cycle of 52.5\% (134 of 255).
    This value is the motor ``deadband''.

    Our policy functions return a unitless floating-point motor output ranging from -1 to 1.

  \subsubsection{Velocity and Acceleration Estimation}

    Paddle velocity is estimated as the discrete difference of low-pass filtered
    angle measurements.
    Similarly, the acceleration is estimated as the discrete difference of
    low-pass filtered velocity estimations.
    Before passing to the policy functions, the acceleration estimations are
    again low-pass filtered.

\section{Discussion and Conclusions}

The Wall mode is implemented as a stiff spring law and viscous damping, which
provides a reasonable sensation of rigidity, although somewhat spongy.
The Wall with Impact Vibration is an attempt to improve the perception of
stiffness by simulating vibrations that would be felt when tapping on a rigid
surface, like a table. We tried several sets of constants according to Okamura
et. al. [2], however, in each case the impact vibrations were not noticable compared
to the simple Wall mode. This may be due to limits of the Hapkit itself. Tests
to measure the frequency response of the Hapkit may illuminate the accuracy
possible.

The Valley and Hillock modes were designed by calculating the gravitational
force that would act on a frictionless rolling object.
Initially we attempted to add mass and intertia to the cursor in the Valley mode
so that the cursor would ``ring'' in and settle in the bottom.
However, the level of filtering necessary to achieve smooth velocity and acceleration
measurements induced noticable delay when using the acceleration estimate to simulate intertia.
When layered on top of other surfaces (like the valley) the delay was enough to cause instability.
We left the intertia simulation in the Flat mode for demonstration, but it does
not convey the feeling of mass, instead just a slow sort of thick liquid damping.
With only the force policy and no intertia, the Valley cursor does ``ring'' due
to the actual intertia and mass of the haptic paddle.

The Notch and Bumpity modes perform well, providing convincing and pleasant feedback.
The force needed to be balanced to prevent either the cursor getting ``stuck''
on the side of a detent or bouncing back and forth against the detent sides.
Occasionally the cursor can stick towards the edges of the range due to
inconsistent fricion in the Hapkit, but that is preferrable to jittering.

\section{References}
\begin{enumerate}
\item http://hapkit.stanford.edu/build.html

\item Allison M Okamura, Mark R Cutkosky, and Jack T Dennerlein. Reality-based models
for vibrationfeedback  in  virtual  environments.IEEE/ASME transactions on
mechatronics,  6(3):245–252, 2001.
\end{enumerate}

\end{document}
